\documentclass{uni_tue_template}

% content of left head area e.g. subject like ETI
\def \headLeft{LEFT}

% content of center head area, used for names
\def \names{ME \& YOU}

% content of right head area e.g. semester like WiSe 2012/13
\def \headRight{RIGHT}

% set name for exercises
\def \exerciseName{exName}

\begin{document}
Dies ist eine Beispiel-Dokument

Wie kann man den \verb+\exercise{}+ Befehl verwenden?\\
Man kann die Nummerierung automatisch von \LaTeX\hspace{1.5mm}übernehmen lassen, indem man die Klammern hinter dem Befehl leer lässt. Also \verb+\exercise{}+.\\
Möchte man eine individuelle Nummerierung festlegen schreibt man in die Klammern hinter dem Befehl die gewünschte Bezeichnung. Also \verb+\exercise{A}+.

Beispiel:\\
\exercise{}
\exercise{}
automatische Nummerierung

\exercise{A}
individuelle Nummerierung

\vspace{10mm}
Wie verwendet man \verb+\subExBegin{}+ und \verb+\subExEnd+ ?\\
Mit \verb+\subExBegin{}+ eröffnet man eine die \verb+subEx+-Umgebung, in den Klammern hinter dem Befehl kann wie in \verb+enumerate+ die Formatierung angegeben werden.

Beispiel:
\subExBegin{}
	\item Einsum
	\subExBegin{}
		\item ium
		\item iium
	\subExEnd
	\item Zweium
\subExEnd

Beispiele für \texttt{Limits}- Befehl:

$\limsum{a}{b}$\\
$\limprod{a}{b}$\\
$\limlim{a\rightarrow\infty}$

\end{document}
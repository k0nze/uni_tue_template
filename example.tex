\documentclass{uni_tue_template}

% content of left head area e.g. subject like ETI
\def \headLeft{LEFT}

% content of center head area, used for names
\def \names{ME \& YOU}

% content of right head area e.g. semester like WiSe 2012/13
\def \headRight{RIGHT}

% set name for exercises
\def \exerciseName{exName}

\begin{document}
\textbf{Dies ist ein Beispiel-Dokument}

Wie kann man den \verb+\exercise{}+ Befehl verwenden?\\
Man kann die Nummerierung automatisch von \LaTeX\hspace{1.5mm}übernehmen lassen, indem man die Klammern hinter dem Befehl leer lässt. Also \verb+\exercise{}+.\\
Möchte man eine individuelle Nummerierung festlegen schreibt man in die Klammern hinter dem Befehl die gewünschte Bezeichnung. Also \verb+\exercise{A}+.

Beispiel:\\
\exercise{}
\exercise{}
automatische Nummerierung

\exercise{A}
individuelle Nummerierung

\vspace{5mm}
Wie verwendet man \verb+\subExBegin{}+ und \verb+\subExEnd+ ?\\
Mit \verb+\subExBegin{}+ eröffnet man eine die \verb+subEx+-Umgebung, in den Klammern hinter dem Befehl kann wie in \verb+enumerate+ die Formatierung angegeben werden.

Beispiel:
\subExBegin{}
	\item Einsum
	\subExBegin{}
		\item ium
		\item iium
	\subExEnd
	\item Zweium
\subExEnd

\vspace{5mm}
Wie kann man Tabellen erstellen?\\
Eine Tabellen-Umgebung wird mit \verb+\begin{table}{FORMAT}+ eröffnet. Bei \verb+FORMAT+ wird wie bei \verb+tabular+ oder \verb+booktabs+ das Tabellenformat angegeben. Die erste Zeile der Tabelle wird mit dem Befehl \verb+\tableHead{}+ umschlossen, hierbei wird am Ende kein Zeilenumbruch mit \verb+\\+ oder \verb+\newline+ benötigt. Der Tabelleninhalt kann wie von \verb+tabular+ oder \verb+booktabs+ gewohnt eingegeben werden. Beendet wird die Tabellenumgebung mit \verb+\end{table}+.

\begin{table}{ccc}
\tableHead{1 & 2 & 3}
a & b & c \\
d & e & f
\end{table}


\vspace{5mm}
\textbf{Befehlsübersicht}

\begin{tabu*}{lll}
\tabucline[1pt]{-}
\tableHead{Name & Befehl & Ausgabe}
UND-Junktor & \verb|\land| & $\land$\\
ODER-Junktor & \verb|\lor| & $\lor$\\
NICHT-Junktor & \verb|\lnot| & $\lnot$\\
XOR-Junktor & \verb|\lxor| & $\lxor$\\ \hline
natürliche Zahlen & \verb|\mn| & $\mn$\\
reelle Zahlen & \verb|\mr| & $\mr$\\
ganze Zahlen & \verb|\mz| & $\mz$\\
rationale Zahlen & \verb|\mq| & $\mq$\\
komplexe Zahlen & \verb|\mc| & $\mc$\\
Lösungsmenge & \verb|\ml| & $\ml$\\
leere Menge & \verb|\mvoid| & $\mvoid$\\
Potenzmenge & \verb|\mpz| & $\mpz$\\ \hline
Vereinigung & \verb|\mor| & $\mor$\\
Schnitt & \verb|\mand| & $\mand$\\
symmetrische Differenz & \verb|\mnand| & $\mnand$\\
Differenzmenge & \verb|\mnot| & $\mnot$\\
Komplement & \verb|\mcp| & $\mcp$\\
Teilmenge & \verb|\mpo| & $\mpo$\\
Summe aller Vereinigungen & \verb|\morsum{UNTEN}{OBEN}| & $\morsum{A\in\mathcal{A}}{n}$\\
Summe aller Schnitte & \verb|\mandsum{UNTEN}{OBEN}| & $\mandsum{A\in\mathcal{A}}{n}$\\ 
Karthesisches Produkt & \verb|\mkth{UNTEN}{OBEN}| & $\mkth{A\in\mathcal{A}}{n}$ \\\hline
build & \verb|\build| & \build \\
credits & \verb|\credits| & \\
\tabucline[1pt]{-}
\end{tabu*}

\end{document}
\documentclass{uni_tue_template}

% content of left head area e.g. subject like ETI
\def \headLeft{LEFT}

% content of center head area, used for names
\def \names{ME \& YOU}

% content of right head area e.g. semester like WiSe 2012/13
\def \headRight{RIGHT}

% set name for exercises
\def \exerciseName{exName}

\begin{document}
\textbf{Dies ist ein Beispiel-Dokument}

\textbf{Wie kann man den} \verb+\exercise{}+ \textbf{Befehl verwenden?}\vspace{2mm}\\
Man kann die Nummerierung automatisch von \LaTeX\hspace{1.5mm}übernehmen lassen, indem man die Klammern hinter dem Befehl leer lässt. Also \verb+\exercise{}+.\\
Möchte man eine individuelle Nummerierung festlegen schreibt man in die Klammern hinter dem Befehl die gewünschte Bezeichnung. Also \verb+\exercise{A}+.

Beispiel:\\
\exercise{}
\exercise{}
automatische Nummerierung

\exercise{A}
individuelle Nummerierung

\vspace{5mm}
\textbf{Wie verwendet man} \verb+\subExBegin{}+ \textbf{und} \verb+\subExEnd+ \textbf{?}\vspace{3mm}\\
Mit \verb+\subExBegin{}+ eröffnet man eine die \verb+subEx+-Umgebung, in den Klammern hinter dem Befehl kann wie in \verb+enumerate+ die Formatierung angegeben werden.

Beispiel:
\subExBegin{}
	\item Einsum
	\subExBegin{}
		\item ium
		\item iium
	\subExEnd
	\item Zweium
\subExEnd

\vspace{3mm}
\textbf{Wie kann man Tabellen erstellen?}\vspace{2mm}\\
Eine Tabellen-Umgebung wird mit \verb+\begin{table}{FORMAT}+ eröffnet. Bei \verb+FORMAT+ wird wie bei \verb+tabular+ oder \verb+booktabs+ das Tabellenformat angegeben. Die erste Zeile der Tabelle wird mit dem Befehl \verb+\tableHead{}+ umschlossen, hierbei wird am Ende kein Zeilenumbruch mit \verb+\\+ oder \verb+\newline+ benötigt. Der Tabelleninhalt kann wie von \verb+tabular+ oder \verb+booktabs+ gewohnt eingegeben werden. Beendet wird die Tabellenumgebung mit \verb+\end{table}+.

\begin{table}{ccc}
\tableHead{1 & 2 & 3}
a & b & c \\
d & e & f
\end{table}
\newpage

\textbf{How to use the \gqm{correction} - command:}

\correction{}{frame}{There once was a young man named Konstantin who was coding all day long until he turned into a compiler himself. This was the day when he, as he likes to call it, "touched the dark side of the force" . From now on, he no longer called himself "Konstantin"\ but "Konze"\ and cruelly began to reject every scientific paper that wasn't written in \LaTeX.}

\correction{.5}{}{There once was a young man named Konstantin who was coding all day long until he turned into a compiler himself. This was the day when he, as he likes to call it, "touched the dark side of the force" . From now on, he no longer called himself "Konstantin"\ but "Konze"\ and cruelly began to reject every scientific paper that wasn't written in \LaTeX.}

\textbf{How to use the \gqm{switch} - command:}

\verb+\CON+
\CON

\verb+\correction{.25}{frame}{You've been a bad boy!}+ turns into \correction{.27}{frame}{You've been a bad boy!}

\verb+\COFF+
\COFF

\verb+\correction{.25}{frame}{You've been a bad boy!}+ turns into \correction{.27}{frame}{You've been a bad boy!} (nada)\vspace{3mm}


\textbf{Funktionen zeichnen}

\begin{minipage}[c]{.4\textwidth}
\begin{function}{ejes=-4:4 -4:4,scale=0.8}
\drawfunction{x^2}
\end{function}
\end{minipage}
\begin{minipage}[c]{.55\textwidth}
\begin{verbatim}
\begin{function}{ejes=-4:4 -4:4,%
scale=0.8}
\drawfunction{x^2}
\end{function}
%
% Optionale Parameter werden durch
% Kommata getrennt und in die Klammer
% nach \begin{function} geschrieben.
% Bsp.:
% \begin{function}{scale=.5,sharp plot}
\end{verbatim}
\end{minipage}
\newpage

\textbf{Die wichtigsten Optionen für Grafiken im Überblick:}

\begin{table}{ll}
\tableHead{Option&Beschreibung}
ejes=\{xmin:xmax ymin:ymax\}&x- und y- Bereich der Grafik\\
scale=\{1.5\}&Vergrößert die Grafik um den Faktor 1.5\\
ticks=none&Keine kleine Striche an den Achsen\\
xticklabels=$\backslash$empty&Keine x- Achsenbeschriftung\\
yticklabels=$\backslash$empty&Keine y- Achsenbeschriftung\\
sharp plot&keine Kantenglättung beim Funktionsplot\\
smooth&Kantenglättung beim Funktionsplot\\
legend Entries=\{fkt1,fkt2,fkt3\}&Erzeugt eine Legende\\
title=\{Titel\}&Schreibt das Wort \gqm{Titel} über die Grafik\\
grid=major&zeichnet ein Gitter\\
domain=-4:4&Berich, in dem Funktionen dargestellt werden\\
samples=100&100 Punkte berechnet, um Funktion zu plotten\\
xlabel=\{$x$\}&x- Achsenbeschriftung\\
ylabel=\{$f(x)$\}&y- Achsenbeschriftung\\
xtick=\{-3,-2.5,...,4\}&x- Intervall von -3 bis 4 in 0,5er Schritten\\
ytick=\{-3,-2.5,...,4\}&y- Intervall von -3 bis 4 in 0,5er Schritten\\
width=5cm&Breite der Grafik\\
height=5cm&Höhe der Grafik
\end{table}

\textbf{Anmerkung:} Es ist auch möglich, die Klammer hinter \verb+\begin{function}+ leer zu lassen, dann wird alles auf default- Werte gesetzt.

\newpage

\textbf{Code einbinden}\vspace{2mm}\\
Dies ist ein Beispiel für den Verwendung der von \verb+\begin{lstlisting}[style=LANG]+ und \verb+\end{lstlisting}+.\vspace{3mm}\\
Beispiel: \textbf{Java}

\begin{lstlisting}[style=java]
public class Test {
	private static final int myStaticNumber = 7;
	public static void main(String[] args) {
		System.out.println("Hello World");
		
		/* This is a very stupid thing to do */
		for (int i = 0; i < 10; ++i) {
			--i;
		}
		
		/* See the source code if you want to write 
		 * someting in italics */
		System.out.println(Test.@@myStaticNumber@@);
	}
}
\end{lstlisting}


Beispiel: \textbf{C++}

\begin{lstlisting}[style=cpp]
#include <iostream>
using namespace std;

int main() {
	cout << "Hello everybody!" << endl;
}
\end{lstlisting}

\begin{lstlisting}[style=mipsasm]
sw $ra,($sp)
addi $sp,$sp,-4
sw $fp,($sp)
addi $sp,$sp,-4
add	$fp,$sp,12
lw $t0,($fp)
li $t1, 2
bgt $t0,$t1,do_recurse
li $t0, 1
\end{lstlisting}
Verfügbare Sprachen: java, php, cpp


\newpage

\vspace{5mm}
\textbf{Befehlsübersicht}

\begin{tabu*}{lll}
\tabucline[1pt]{-}
\tableHead{Name & Befehl & Ausgabe}
UND-Junktor & \verb|\land| & $\land$\\
ODER-Junktor & \verb|\lor| & $\lor$\\
NICHT-Junktor & \verb|\lnot| & $\lnot$\\
XOR-Junktor & \verb|\lxor| & $\lxor$\\
\hline
natürliche Zahlen & \verb|\mn| & $\mn$\\
reelle Zahlen & \verb|\mr| & $\mr$\\
ganze Zahlen & \verb|\mz| & $\mz$\\
rationale Zahlen & \verb|\mq| & $\mq$\\
komplexe Zahlen & \verb|\mc| & $\mc$\\
Lösungsmenge & \verb|\ml| & $\ml$\\
leere Menge & \verb|\mvoid| & $\mvoid$\\
Potenzmenge & \verb|\mpz| & $\mpz$\\
\hline
Vereinigung & \verb|\mor| & $\mor$\\
Schnitt & \verb|\mand| & $\mand$\\
symmetrische Differenz & \verb|\mnand| & $\mnand$\\
Differenzmenge & \verb|\mnot| & $\mnot$\\
Komplement & \verb|\mcp| & $\mcp$\\
Teilmenge & \verb|\mpo| & $\mpo$\\
Summe aller Vereinigungen & \verb|\morsum{UNTEN}{OBEN}| & $\morsum{A\in\mathcal{A}}{n}$\\
Summe aller Schnitte & \verb|\mandsum{UNTEN}{OBEN}| & $\mandsum{A\in\mathcal{A}}{n}$\\ 
Karthesisches Produkt & \verb|\mkth{UNTEN}{OBEN}| & $\mkth{A\in\mathcal{A}}{n}$ \\
\hline
Summe/Reihe & \verb|\limsum{UNTEN}{OBEN}| & $\limsum{n=0}{\infty}$\\
Produkt & \verb|\limprod{UNTEN}{OBEN}| & $\limprod{n=0}{\infty}$
\end{tabu*}

\newpage

\begin{tabu*}{lll}
\tabucline[1pt]{-}
\tableHead{Name & Befehl & Ausgabe}
Limes & \verb|\limlim{UNTEN}| & $\limlim{n\rightarrow\infty}$\\
\tabucline[.5pt]{-}
Blitz- Symbol&\verb|\lightning|&\\
\hline
Grad einer Funktion & \verb|\grad| & $\grad$ \\ 
Umkehrfunktion & \verb|\ifu{f}| & $\ifu{f}$\\ 
\gqm{differenzierbar} & \verb|\diffbar| & \diffbar\\
Differentialquotient & \verb|\dq{x}| & $\dq{x}$\\
Integral & \verb|\limint{UNTEN}{OBEN}| & $\limint{a}{b}$\\
dx & \verb|\intd{x}| & $\intd{x}$\\
\hline
runde Klammern & \verb|\lrr{INHALT}| & $\lrr{\dfrac{1}{5}}$ \\
eckige Klammern & \verb|\lra{INHALT}| & $\lra{\dfrac{1}{2}}$ \\
geschweifte Klammern &\verb|\lrc{INHALT}|& $\lrc{\dfrac{1}{3}}$ \\
Absolutbetrag & \verb|\lrabs{INHALT}| & $\lrabs{\dfrac{1}{4}}$ \\
Floor & \verb|\floor{INHALT}| & $\floor{\dfrac{x}{x^2}}$ \\
Ceiling & \verb|\ceiling{INHALT}| & $\ceiling{\dfrac{1}{x}}$ \\
\hline
deutsche Anführungszeichen & \verb|\gqm{}| & \gqm{Hallo} \\
build & \verb|\build| & \build \\
credits & \verb|\credits| & \\
\tabucline[1pt]{-}
\end{tabu*}


\end{document}